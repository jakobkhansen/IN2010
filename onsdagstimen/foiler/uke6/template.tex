\documentclass[norsk]{beamer}
%\usepackage[utf8]{inputenc}
\usepackage{listings}
\usepackage{color}
\usepackage[norsk]{babel}
\usepackage{hyperref}

\definecolor{pblue}{rgb}{0.2,0.2,0.7}
% \definecolor{pgreen}{rgb}{0,0.5,0}
\definecolor{pred}{rgb}{0.7,0.2,0.2}
\definecolor{pgrey}{rgb}{0.46,0.45,0.48}

\lstset{
	language=Java,
	numbers=left,
	breaklines=true,
	tabsize=4,
	commentstyle=\color{pgrey},
	keywordstyle=\color{pblue},
	stringstyle=\color{pred},
    showstringspaces=false,
	literate={\ \ }{{\ }}1
}

%Information to be included in the title page:
\title{IN2010 uke 6}
\author{Jakob Hansen}
\date{\today}

\begin{document}
	\frame{\titlepage}
	\begin{frame}
       \frametitle{Ting vi kan snakke om i dag}
       \begin{itemize}
           \item Repetisjon
		   \item Korteste sti
			   \begin{itemize}
				   \item Dijkstra
				   \item Bellman-Ford
				\end{itemize}
		   \item Minimale spenntrær
			   \begin{itemize}
				   \item Prims
				   \item Kruskals
				   \item Boruvka
				\end{itemize}
       \end{itemize}
    \end{frame}

	\begin{frame}{Kort recap om grafer}
		\begin{itemize}
			\item Grafer er bare noder og kanter! (\{V\}, \{E\})
			\item Rettet, urettet
			\item Vektet, uvektet
			\item Komponenter, sammenhengende
		\end{itemize}
	\end{frame}


	\begin{frame}{Shortest path}
		\begin{columns}
			\column{0.5\textwidth}
		\begin{itemize}
			\item Veldig kjent problem innen grafteori.
			\item Finne stien mellom 2 noder med minimal vekt.
		\end{itemize}

			\column{0.5\textwidth}
			\includegraphics[width=150px]{shortestpathfirst.png}
		\end{columns}
	\end{frame}

	\begin{frame}{Dijkstra}
		\begin{itemize}
			\item Kanskje den mest kjente grafalgoritmen
			\item Grådig algoritme -> Fungerer ikke alltid, men rask!
			\item Bruker en prioritetskø til å avgjøre korteste sti fra en node til alle
				andre noder.
		\end{itemize}
	\end{frame}

	\begin{frame}{Pseudokode og kompleksitet}
		\includegraphics[width=250px]{dijkstrapseudo.png}
	\end{frame}

	\begin{frame}{Bellman-Ford}
		\begin{itemize}
			\item Enklere algoritme, som fungerer på negative vekter!
			\item Treigere :(
		\end{itemize}
		\includegraphics[width=250px]{bellmanfordpseudo.png}
	\end{frame}

	\begin{frame}{Minimale spenntrær}
		\begin{itemize}
			\item Spenntre -> Minst antall kanter som fortsatt er en sammenhengende graf
			\item Minimalt spenntre -> Spenntreet med minst sum av vektede kanter
			\item Vi skal se på 3 algoritmer, alle kjører i O(|E|*log(|V|))
		\end{itemize}
	\end{frame}

	\begin{frame}{Prims}
		\begin{itemize}
			\item Likner ganske på Dijkstra!
			\item Ta den minste kanten som forbinder en ny node til spenntreet, til vi er
				ferdig
		\end{itemize}
	\end{frame}

	\begin{frame}{Kruskals}
		\begin{itemize}
			\item Bygger mange "spennskoger" og setter de sammen, helt til det bare er et
				spenntre
			\item Legger til den minste kanten hver gang
		\end{itemize}
	\end{frame}

	\begin{frame}{Boruvka}

	\begin{itemize}
		\item Likner på Kruskals, bedre egnet for parallellisering
		\item Kombinerer komponenter helt til det bare er 1.
		\item Hver komponent velger den minste utgående kanten i hver iterasjon.
	\end{itemize}

	\end{frame}

\end{document}

