\documentclass[norsk, handout]{beamer}
%\usepackage[utf8]{inputenc}
\usepackage{listings}
\usepackage{color}
\usepackage[norsk]{babel}
\usepackage{fontspec}


\definecolor{pblue}{rgb}{0.2,0.2,0.7}
% \definecolor{pgreen}{rgb}{0,0.5,0}
\definecolor{pred}{rgb}{0.7,0.2,0.2}
\definecolor{pgrey}{rgb}{0.46,0.45,0.48}

\lstset{
	language=Java,
	numbers=left,
	breaklines=true,
	tabsize=4,
	commentstyle=\color{pgrey},
	keywordstyle=\color{pblue},
	stringstyle=\color{pred},
    showstringspaces=false,
	literate={\ \ }{{\ }}1
}

%Information to be included in the title page:
\title{IN2010 uke 5}
\author{Jakob Hansen}
\date{\today}

\begin{document}
	\frame{\titlepage}
    \begin{frame}
		\frametitle{Ting vi kan snakke om i dag}
		\begin{itemize}
			\item Repetisjon
			\item Grafer (Masse begreper!)
			\item Traversering av grafer
			\item Topologisk sortering
		\end{itemize}
    \end{frame}

	\begin{frame}
		\frametitle{Motivasjon bak grafer}

		\begin{itemize}
			\item Hva er grafer? Hva brukes de til?
				\pause
			\item Veldig generalisert datatype, kan brukes til å løse utrolig mange
				problemer, lett å representere
			\item Korteste sti, datanettverk flyt, kritiske enheter i systemer, ...
		\end{itemize}
		\begin{center}
			\includegraphics[width=200px]{rugraph.png}
		\end{center}
	\end{frame}

	\begin{frame}{Definisjon av grafer}
			\begin{columns}[onlytextwidth,T]
				\column{0.45\textwidth}
				\includegraphics[width=125px]{undirectedgraph.png}
				\includegraphics[width=125px]{directedgraph.png}

				\column{0.55\textwidth}
				\begin{itemize}
					\item En mengde noder og en mengde kanter (V, E)
						\pause
					\item Kombinasjoner av egenskaper:
					\item Rettet, urettet
						\pause
					\item Vektet, uvektet
						\pause
					\item simpel, multigraf
						\pause
					\item Syklisk, asyklisk
						\pause
					\item Komponenter, sammenhengende
						\pause
					\item Stier, veier
						\pause
					\item Inngrad, utgrad
				\end{itemize}
			\end{columns}

	\end{frame}

	\begin{frame}{Representere grafer}
		\begin{itemize}
			\item Mange måter å representere grafer på, mange problemer kan representeres
				som grafer
			\item Nabolister
			\item Nabomatrise
			\item Operasjoner får forskjellig kompleksitet avhengig av representasjon
		\end{itemize}
	\end{frame}

	\begin{frame}{Traversering av grafer}
		\begin{itemize}
			\item Reise fra en node til en annen gjennom kanter
			\item Hvor mange noder kan vi nå? Er grafen sammenhengende? ...
			\item Brukes omtrent i alle andre grafalgoritmer!
			\item Forskjellige strategier -> DFS og BFS
		\end{itemize}
	\end{frame}

	\begin{frame}{DFS, dybde først søk}
		\begin{itemize}
			\item Reis så dypt som mulig i en retning, så en annen, osv.
			\item Kan gjøres med en vanlig stack, men enklere med rekursjonsstacken
			\item I et tre ville man gått så dypt ned i en branch som mulig først
			\item Enkel å implementere, bruker lite minne
		\end{itemize}
	\end{frame}

	\begin{frame}{BFS, bredde først søk}
		\begin{itemize}
			\item Besøk alle noder som er 1 kant unna start, så 2, 3, ... lagvis
			\item Reiser fra sentrum og utover i alle retninger
			\item Bruker en kø, legg til alle noder i neste steg på køen mens du ser på
				dette steget
			\item I et tre, ville gått lagvis nedover i høyden
		\end{itemize}
	\end{frame}

	\begin{frame}{Topologisk sortering}
		\begin{columns}
			\column{0.5\textwidth}
			\includegraphics[width=150px]{topsort.png}

			\column{0.6\textwidth}
			\begin{footnotesize}
			\begin{itemize}
				\item Bruker DAG's for å representere avhengigheter
				\item Topologisk sortering: lovlig rekkefølge av "utføring"
					\pause
				\item Algoritme: Legg alle noder med inngrad 0 i en kø, ta ut en node av
					køen og fjern kantene ut fra den, repeter til grafen er tom.
					\pause
				\item Hvis vi har flere noder igjen, og vi ikke kan ta ut en node med
					inngrad 0, hva betyr det?
			\end{itemize}
		\end{footnotesize}

		\end{columns}
	\end{frame}
\end{document}

