\documentclass[norsk]{beamer}

%\usepackage[utf8]{inputenc}
\usepackage{listings}
\usepackage{color}
\usepackage[norsk]{babel}

\definecolor{pblue}{rgb}{0.2,0.2,0.7}
% \definecolor{pgreen}{rgb}{0,0.5,0}
\definecolor{pred}{rgb}{0.7,0.2,0.2}
\definecolor{pgrey}{rgb}{0.46,0.45,0.48}

\lstset{
	language=Java,
	numbers=left,
	breaklines=true,
	tabsize=4,
	commentstyle=\color{pgrey},
	keywordstyle=\color{pblue},
	stringstyle=\color{pred},
    showstringspaces=false,
	literate={\ \ }{{\ }}1
}

%Information to be included in the title page:
\title{IN2010 halvveis recap}
\author{Jakob Hansen}
\date{\today}

\begin{document}
	\frame{\titlepage}
	\begin{frame}{Hva vi skal snakke om i dag}
		\begin{itemize}
			\item Recap av pensum til nå! :)
		\end{itemize}
    \end{frame}

	\begin{frame}{Uke 35}
		\begin{itemize}
			\item Big O
				\begin{itemize}
					\item Analyserer kjøretid for et program.
					\item Abstraherer bort "små" forskjeller, ser på den generelle
						veksten.
					\item Gjør det lettere å snakke om kjøretid og sammenlikne programmer
				\end{itemize}
			\item Binærsøk
				\begin{itemize}
					\item Algoritme som finner indeksen til et element i et sortert array
						veldig raskt.
					\item Se på midten, hvis elementet man ser etter er mindre, gå til
						venstre halvdel, hvis det er større, gå til høyre halvdel.
						Repeat.
					\item $O(log(n))$
				\end{itemize}
		\end{itemize}
		\begin{center}
			\includegraphics[width=200px]{binarysearch.jpeg}
		\end{center}
	\end{frame}

	\begin{frame}{Uke 35}
			\begin{itemize}
				\item Trær
					\begin{itemize}
						\item Lagrer verdier med en key som plasserer mindre keys til
							venstre, større til høyre
						\item Innsetting: Reis nedover treet til du finner en nullpeker.
							Reis til venstre dersom key er mindre, høyre hvis større.
						\item Sletting: 3 caser: 0-2 barn.
							\begin{itemize}
								\item 0 barn: bare fjern pekeren fra forelder
								\item 1 barn: Erstatt noden med barnet
								\item 2 barn: Reduser til case 0/1 ved å erstatte noden med
									neste node i inorder traversal.
							\end{itemize}
					\end{itemize}
			\end{itemize}

			\begin{center}
				\includegraphics[width=150px]{bst.png}
			\end{center}
	\end{frame}

	\begin{frame}{Uke 36}
		\begin{itemize}
			\item Balanserte søketrær
				\begin{itemize}
					\item Balanserer seg selv, for å få en garantert kortere høyde
					\item Da kan vi gjøre innsetting, sletting osv i $O(log(n))$ tid worst
						case
					\item Retter opp i treet gjennom rotasjoner
				\end{itemize}
			\item AVL trær
			\item Rødsvarte trær
		\end{itemize}
	\end{frame}
	\begin{frame}{Uke 36}
		\begin{itemize}
			\item AVL trær
			\begin{itemize}
				\item Høyde: max(høyre.høyde, venstre.høyde) + 1
				\item Balanse: høyre.høyde - venstre.høyde
				\item Balanseverdien b må være $-1 \leq b \leq 1$
				\item Innsetting: Sett inn som vanlig, oppdater høyde og balanse på
					tilbakeveien i rekursjonsstacken. Hvis balanseverdien ikke er
					riktig, gjør rotasjoner i retningen av ubalansen for å fikse det!
				\item Sletting: Slett som vanlig. Gjør det samme som innsetting på
					tilbakeveien!
			\end{itemize}

			\begin{center}
				\includegraphics[width=200px]{avltree.png}
			\end{center}
		\end{itemize}
	\end{frame}
	\begin{frame}{Uke 36}
		\begin{itemize}
			\item Rødsvarte trær

			\begin{itemize}
				\item Ganske abstrakt implementasjon
				\item Regler:
					\begin{itemize}
						\item Alle noder er røde eller svarte
						\item Hvis en node er rød, så er barna svarte
						\item Det må være like mange svarte noder på stien fra rot til alle
							nullpekere
						\item Ekstra: nullpekere er sorte, roten er sort
					\end{itemize}
				\item Rødsvarte trær gjør færre rotasjoner enn AVL, men er litt mindre
					balansert
			\end{itemize}
		\end{itemize}

		\begin{center}
			\includegraphics[width=170px]{redblack.png}
		\end{center}
	\end{frame}

	\begin{frame}{Uke 37}
	\begin{itemize}
		\item Prioritetskø
			\begin{itemize}
				\item abstrakt datatype
				\item Har metoder som removeMin, insert, size, osv...
			\end{itemize}
	\end{itemize}
	\end{frame}

	\begin{frame}{Uke 37}

		\begin{itemize}
			\item Heap (binær heap)
				\begin{itemize}
					\item Implementasjon av prioritetskø
					\item MinHeap -> Barna til en node må ha større verdi
					\item MaxHeap -> Barna til en node må ha mindre verdi
					\item Da er alltid det minste/største elementet øverst
					\item Innsetting: Sett inn nederst til venstre, boble opp om
						nødvendig.
					\item Sletting: Erstatt roten med elementet nederst til venstre,
						boble ned om nødvendig.
					\item Kompleksitet på innsetting og sletting: $O(log(n))$
				\end{itemize}
		\end{itemize}

		\begin{center}
			\includegraphics[width=200px]{heap.jpg}
		\end{center}
	\end{frame}

	\begin{frame}{Uke 37}
		\begin{itemize}
			\item Huffman koding

			\begin{itemize}
				\item Mål: Gjøre ``encodingen`` av en streng så kort som mulig + uniquely
					decodable
				\item Løsning: La bokstavene som fremkommer oftest ha kort bit
					representasjon.
				\item Implementasjon: Prioritetskø med alle bokstavene der prioriteten er
					antall forekomster. Bygg et binært tre fra bunnen og opp.
				\item Encodingen til en bokstav: Venstre er 0, høyre er 1, reis nedover
					treet fra roten for å finne encodingen.
				\item Uniquely decodable: Ingen encodinger er en prefiks av en annen
					encoding.
			\end{itemize}
		\end{itemize}
	\end{frame}

	\begin{frame}{Uke 37}
		\begin{center}
			\includegraphics[width=300px]{huffmantree.png}
		\end{center}
	\end{frame}

	\begin{frame}{Uke 38}
		\begin{itemize}
			\item Grafer
				\begin{itemize}
					\item Noder og kanter
					\item Rettet / urettet
					\item Vektet / uvektet
					\item Sammenhengende, komponenter
				\end{itemize}
		\end{itemize}
			\begin{center}
				\begin{columns}
					\column{0.5\textwidth}
				\includegraphics[width=150px]{rugraph.png}
					\column{0.5\textwidth}
				\includegraphics[width=150px]{directedgraph.png}
			\end{columns}
		\end{center}
	\end{frame}

	\begin{frame}{Uke 38}
		\begin{itemize}
			\item Traversering
			\item DFS
				\begin{itemize}
					\item Gå så dypt som mulig i en retning, så prøv en annen
					\item Bruker en stack (ofte rekursjonsstacken)
				\end{itemize}
			\item BFS
				\begin{itemize}
					\item Ta alle noder 1 kant ut fra start, så 2 kanter, 3, 4, ...
					\item Bruker en kø (FIFO)
				\end{itemize}
		\end{itemize}
	\end{frame}

	\begin{frame}{Uke 39}
		\begin{itemize}
			\item Korteste sti
				\begin{itemize}
					\item Finne korteste sti (minimal vekt) mellom 2 noder (alternativt mellom alle
						noder)
					\item Dijkstra, Bellman-Ford
				\end{itemize}
			\item Minimale spenntrær
				\begin{itemize}
					\item Minimalt spenntre = minimal sammenhengende subgraf med minimal
						sammenlagt kost.
					\item Prims, Kruskals, Boruvka
				\end{itemize}
			\item Baserer seg alle på en liknende ide: Legg noder/kanter i en
				prioritetskø, ta ut og gå over kanter. Får
				kompleksitet $O(|E|*log(|V|))$ (Utenom Bellman-Ford)
		\end{itemize}
	\end{frame}

	\begin{frame}{Uke 39}
		\begin{itemize}
			\item Dijkstra
				\begin{itemize}
					\item Lag en prioritetskø der alle noder utenom start har prioritet
						$\infty$, start har 0. Dette kalles ``relax`` verdien.
					\item Loop: Ta ut minste element, se på alle kanter ut, hvis
						kantnoden er i køen, og relaxverdien til noden du tok ut + kosten
						av kanten er mindre enn relaxverdien til kanten, oppdater
						relaxverdien.
				\end{itemize}
			\item Bellman-Ford
				\begin{itemize}
					\item Prøv alle stier |V| - 1 kanter i lengde ut fra startnoden!
					\item Ha en liknende prioritetskø som Dijkstra.
					\item Gå over alle kanter i grafen |V| - 1 ganger, oppdater relax verdi.

				\end{itemize}
			\item Bruk et predecessor array/instansvariabel for å finne den
				korteste stien
		\end{itemize}
	\end{frame}

	\begin{frame}{Uke 39}
		\begin{itemize}
			\item Prims
				\begin{itemize}
					\item Noe likt Dijkstra, start fra en node, bygg opp et tre ved å
						legge til den billigste kanten ut fra treet.
				\end{itemize}
			\item Kruskals
				\begin{itemize}
					\item Legg alltid til den billigste kanten!
					\item Enkel tanke, verre å programmere
				\end{itemize}
			\item Boruvka
				\begin{itemize}
					\item Noe lik Kruskals, la hver node være en komponent.
					\item Loop: Hver eneste komponent legger til en kant ut fra
						komponenten. Halverer antall komponenter i hver iterasjon.
					\item Godt egnet for parallellisering, men har samme kompleksitet!
				\end{itemize}
		\end{itemize}
	\end{frame}

\end{document}

