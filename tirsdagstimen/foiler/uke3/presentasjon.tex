\documentclass[norsk, handout]{beamer}
\usepackage[utf8]{inputenc}
\usepackage{listings}
\usepackage{color}
\usepackage[norsk]{babel}

\definecolor{pblue}{rgb}{0.2,0.2,0.7}
% \definecolor{pgreen}{rgb}{0,0.5,0}
\definecolor{pred}{rgb}{0.7,0.2,0.2}
\definecolor{pgrey}{rgb}{0.46,0.45,0.48}

\lstset{
	language=Java,
	numbers=left,
	breaklines=true,
	tabsize=4,
	commentstyle=\color{pgrey},
	keywordstyle=\color{pblue},
	stringstyle=\color{pred},
    showstringspaces=false,
	literate={\ \ }{{\ }}1
}

%Information to be included in the title page:
\title{IN2010 uke 3}
\author{Jakob Hansen \\ \texttt{jakobkha@uio.no}}
\date{\today}

\begin{document}
	\frame{\titlepage}
    \begin{frame}
        \frametitle{Ting vi kan snakke om i dag}

        \begin{itemize}
            \item Repetisjon
            \item Balanserte trær
                \begin{itemize}
                    \item AVL trær
                    \item Rødsvarte trær
                \end{itemize}
            \item Obligen
        \end{itemize}
    \end{frame}

    \begin{frame}
        \frametitle{Balanserte trær}
        \begin{itemize}
            \item Hva er et balansert tre?
                \pause
            \item Et tre der høyden er relativt lav i forhold til antall noder.
                \pause
            \item Hvorfor ønsker vi balanserte trær?
                \pause
            \item For eksempel for raskere søk og innsetting.
        \end{itemize}
    \end{frame}

    \begin{frame}
        \frametitle{AVL trær}
        \begin{itemize}
            \item Selvbalanserende binært søketre
            \item Hver node har en høydeverdi og en balanseverdi
            \item hoyde = max(venstrenode.hoyde, hoyrenode.hoyde) + 1
            \item hoyde(null) = -1
            \item balanse = venstrenode.hoyde - hoyrenode.hoyde
            \item balanse b til enhver node må være -1 $\leq$ b $\leq$ 1
        \end{itemize}
    \end{frame}

    \begin{frame}
        \frametitle{Innsetting i AVL trær}
        \begin{itemize}
            \item Sett inn i et AVL tre slik som et vanlig binært søketre.
            \item Oppdater høyde på vei oppover i rekursjonsstacken, og sjekk balanse.
            \item Hvis balansen er større enn 1 eller mindre enn -1, må vi fikse treet.
            \item Fiks treet ved å gjøre rotasjoner i forhold til der treet er ``tyngre``
        \end{itemize}
    \end{frame}

    \begin{frame}
        \frametitle{Rotasjoner}
        \begin{center}
            \includegraphics[width=150px]{rotations.png}
        \end{center}
        \begin{center}
            $\Downarrow$
        \end{center}
        \begin{center}
            \includegraphics[width=90px]{finished_rotation.png}
        \end{center}
    \end{frame}

    \begin{frame}
        \frametitle{Sletting i AVL trær}

        \begin{itemize}
            \item Nesten helt likt
            \item Slett som i et vanlig binært søketre
            \item Oppdater høyde på vei oppover i rekursjonsstacken og sjekk balanse
            \item Hvis balansen er større enn 1 eller mindre enn -1, må vi fikse treet.
            \item Gjør rotasjoner i forhold til der treet er ``tyngre``
        \end{itemize}
    \end{frame}

    \begin{frame}
        \frametitle{Rødsvarte trær}
        \begin{itemize}
            \item Ikke grundig gjennomgang!
            \item Regler:
                \begin{itemize}
                    \item Alle noder er enten svarte eller røde
                    \item Hvis en node er rød, så er barna svarte
                    \item Alle stier fra roten til en nullpeker må inneholde like mange
                        svarte noder
                    \item Ekstra: Roten er svart, nullpekere teller som svarte
                \end{itemize}
        \end{itemize}
    \end{frame}
\end{document}

