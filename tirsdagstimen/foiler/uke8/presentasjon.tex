\documentclass[norsk, handout]{beamer}
%\usepackage[utf8]{inputenc}
\usepackage{listings}
\usepackage{color}
\usepackage[norsk]{babel}

\definecolor{pblue}{rgb}{0.2,0.2,0.7}
% \definecolor{pgreen}{rgb}{0,0.5,0}
\definecolor{pred}{rgb}{0.7,0.2,0.2}
\definecolor{pgrey}{rgb}{0.46,0.45,0.48}

\lstset{
	language=Java,
	numbers=left,
	breaklines=true,
	tabsize=4,
	commentstyle=\color{pgrey},
	keywordstyle=\color{pblue},
	stringstyle=\color{pred},
    showstringspaces=false,
	literate={\ \ }{{\ }}1
}

%Information to be included in the title page:
\title{IN2010 uke 8}
\author{Jakob Hansen}
\date{\today}

\begin{document}
	\frame{\titlepage}
	\begin{frame}{Hva vi kan snakke om idag}
		\begin{itemize}
			\item Repetisjon
			\item Noen nye begreper for grafer
			\item Sære algoritmer
				\begin{itemize}
					\item Finne separasjonsnoder i en 2-sammenhengende graf (Bikonnektivitet)
					\item Finne sterkt sammenhengende komponenter
				\end{itemize}
			\item Se på en større eksamensoppgave
		\end{itemize}
    \end{frame}

	\begin{frame}{Litt nye begreper}
		\begin{itemize}
			\item Sammenhengende
			\item k-sammenhengende graf og bikonnektivitet
			\item Separasjonsnoder
			\item Sterkt sammenhengende komponenter
		\end{itemize}
	\end{frame}

	\begin{frame}{Finne ut om en graf er 2-sammenhengende}
		\begin{itemize}
			\item Er grafen fortsatt sammenhengende om en node fjernes?
			\item Naiv algoritme: Fjern en node, sjekk om grafen fortsatt er sammenhengende -> $O(|V|*(|V| + |E|)$
			\item Bedre algoritme "Hopcroft-Tarjan"
				\begin{itemize}
					\item Gjør DFS, sjekk om det er mulig å komme ``tilbake`` til grafen en annen
						vei enn der DFS kom fra
					\item Bruker indekser og ``low-verdier``.
					\item $O(|V| + |E|)$
				\end{itemize}
		\end{itemize}
	\end{frame}

	\begin{frame}{Finne sterkt sammenhengende komponenter}
		\begin{itemize}
			\item A -> B, B -> A
			\item Hvordan finne ut om det finnes en sti fra en node A til alle andre?
				\pause
			\item DFS
			\item Hvordan finne ut om det finnes en sti mellom alle andre noder til A?
				\pause
			\item Reverser grafen og kjør DFS fra A igjen!
			\item Alle noder som kan nås i begge retninger, er sterkt sammenhengende
		\end{itemize}
	\end{frame}

	\begin{frame}{Tarjan's lineær tid algoritme for SCC}
	\begin{itemize}
		\item Kanskje et tilfelle av ``ingen skjønner hvorfor dette fungerer``
		\item Fremgangsmåte:
			\begin{itemize}
				\item Kjør DFS på alle noder, legg alle noder på en stack
				\item Reverser grafen
				\item Kjør DFS igjen, med stack rekkefølgen.
				\item Alle noder som kan nås hver gang man kaller på DFS den andre gangen, er i samme SCC.
			\end{itemize}
	\end{itemize}
	\end{frame}

	\begin{frame}{Eksamensoppgave}
		\begin{center}
			\includegraphics[width=275px]{eksamenintro.png}
		\end{center}
	\end{frame}

	\begin{frame}{Eksempelgraf}
		\begin{center}
			\includegraphics[width=275px]{eksempelgraf.png}
		\end{center}
	\end{frame}

	\begin{frame}{2a}
		\begin{center}
			\includegraphics[width=250px]{2a.png}
		\end{center}
	\end{frame}

	\begin{frame}{2b}
		\begin{center}
			\includegraphics[width=275px]{2b.png}
		\end{center}
	\end{frame}

	\begin{frame}{2c}
		\begin{center}
			\includegraphics[width=275px]{2c.png}
		\end{center}
	\end{frame}

	\begin{frame}{2d}
		\begin{center}
			\includegraphics[width=275px]{2d.png}
		\end{center}
	\end{frame}

	\begin{frame}{2e}
		\begin{center}
			\includegraphics[width=275px]{2e.png}
		\end{center}
	\end{frame}

	\begin{frame}{2f}
		\begin{center}
			\includegraphics[width=250px]{2f.png}
		\end{center}
	\end{frame}

	\begin{frame}{2g}
		\begin{center}
			\includegraphics[width=275px]{2g.png}
		\end{center}
	\end{frame}

	\begin{frame}{2h}
		\begin{center}
			\includegraphics[width=275px]{2h.png}
		\end{center}
	\end{frame}

	\begin{frame}{2i}
		\begin{center}
			\includegraphics[width=275px]{2i.png}
		\end{center}
	\end{frame}
\end{document}

