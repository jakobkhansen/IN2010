\documentclass[norsk, handout]{beamer}
\usepackage[utf8]{inputenc}
\usepackage{listings}
\usepackage{color}
\usepackage[norsk]{babel}
\usepackage{fontspec}

\newfontfamily\Emoji{Twemoji}[Renderer=HarfBuzz]

\definecolor{pblue}{rgb}{0.2,0.2,0.7}
% \definecolor{pgreen}{rgb}{0,0.5,0}
\definecolor{pred}{rgb}{0.7,0.2,0.2}
\definecolor{pgrey}{rgb}{0.46,0.45,0.48}

\lstset{
	language=Java,
	numbers=left,
	breaklines=true,
	tabsize=4,
	commentstyle=\color{pgrey},
	keywordstyle=\color{pblue},
	stringstyle=\color{pred},
    showstringspaces=false,
	literate={\ \ }{{\ }}1
}

%Information to be included in the title page:
\title{IN2010 uke 1-2}
\author{Jakob Hansen \\ \texttt{jakobkha@uio.no}}
\date{\url{https://github.com/jakobkhansen/IN2010}}

\begin{document}
	\frame{\titlepage}
    \begin{frame}
        \frametitle{Dagens plan}
        \center
        \begin{itemize}
            \item Bli litt kjent, corona info, gruppetimens struktur
            \item Diskutere temaet algoritmer og datastrukturer
                \item Big O notasjon + oppgaver
            \item Binærsøk {\Emoji 🔎} + Evt livekode binærsøk
            \item Trær {\Emoji 🌲} + Evt livekode binært tre
            \item Mine tips for å lykkes i IN2010
        \end{itemize}
    \end{frame}

    \begin{frame}
        \begin{center}
            \includegraphics[keepaspectratio, width=300pt, height=\paperheight]{2020-08-24-130904_918x573_scrot.png}
        \end{center}
    \end{frame}

    \begin{frame}
        \begin{center}
            \includegraphics[keepaspectratio, width=300pt,
            height=\paperheight]{2020-08-24-131452_915x573_scrot.png}
        \end{center}
    \end{frame}

    \begin{frame}
        \frametitle{Diskusjoner}
        \begin{itemize}
            \item Hva er en algoritme?
            \pause
            \item Hva er en datastruktur?
            \pause
        \item Hvordan henger algoritmer og datastrukturer sammen? {\Emoji 🤝}
        \end{itemize}
    \end{frame}

    \begin{frame}[fragile]
        \frametitle{Big O notasjon}
        \begin{itemize}
            \item Hva er målet med Big O?
            \pause
        \item Analysere kjøretid! Hvilken algoritme er raskest? (Grovt)
        \pause
        \item Abstrahere bort de små forskjellene
        \pause
        \begin{lstlisting}[basicstyle=\scriptsize]
            boolean numExists(int[] array, int numToFind) {

                // Itererer over array, hvis vi finner tallet, true
                for (int i = 0; i < array.length; i++) {
                    if (array[i] == numToFind) {
                        return true;
                    }
                }

                // Ingen elementer er lik tallet, false
                return false;
            }
		\end{lstlisting}
        \end{itemize}

    \end{frame}

    \begin{frame}
        \frametitle{Big O notasjon}
        \includegraphics[keepaspectratio, width=300pt, height=300pt]{big-o-chart.png}
    \end{frame}

    \begin{frame}
        \frametitle{Konstant tid}
        \begin{itemize}
            \item O(1)
            \item Tar samme tid uansett
            \item Eksempel: Hente første element av et array
            \item O(1000) = O(1)
        \end{itemize}
    \end{frame}

    \begin{frame}
        \frametitle{Lineær tid}

        \begin{itemize}
            \item O(n)
            \item Vokser direkte med input størrelse
            \item Eksempel: Iterere og printe ut hvert element i et array
            \item O(100n) = O(n)
        \end{itemize}

    \end{frame}

    \begin{frame}
        \frametitle{Polynomiell tid}

        \begin{itemize}
            \item For hver n, for hver n...
            \item O($n^x$), for eksempel O($n^2$)
            \item Eksempel: To løkker som sjekker om det finnes en duplikat i arrayet.
            \item Eksempel: ``Bruteforce`` en kodelås
        \end{itemize}
    \end{frame}

    \begin{frame}
        \frametitle{Logaritmisk tid}

        \begin{itemize}
            \item O(log(n))
            \item Litt tricky
            \item $log_2(n) = x$ hvis $2^x = n$
            \item Antall ganger du må halvere n for å få 1
            \item Eksempel: Lete i telefonbok, Binary search
        \end{itemize}
    \end{frame}

    \begin{frame}[fragile]
        \begin{center}
            \includegraphics[width=90pt]{bigo_oppgaver1.png}
            \includegraphics[width=90pt]{bigo_oppgave4.png}

        \begin{lstlisting}[basicstyle=\scriptsize]
                for (int i = n; i > 1; i = i/2) {
                    System.out.println(i);
                }
		\end{lstlisting}
        \end{center}
    \end{frame}

    \begin{frame}[fragile]
        \frametitle{Søk}
        Hvordan finne ut om et element eksisterer i en array?
        \pause
        \begin{lstlisting}[basicstyle=\scriptsize]
            // Den enkle måten å sjekke om et array inneholder et tall
            public boolean numExists(int[] array, int numToFind) {

                // Itererer over array, hvis vi finner tallet, true
                for (int i = 0; i < array.length; i++) {
                    if (array[i] == numToFind) {
                        return true;
                    }
                }

                // Ingen elementer er lik tallet, false
                return false;
            }
		\end{lstlisting}

        \begin{itemize}
            \item Er det mulig å gjøre dette raskere hvis arrayet er sortert?
            \pause
        \item Binary Search {\Emoji🤯}
        \end{itemize}
    \end{frame}

    \begin{frame}
        \frametitle{Binary Search}

        \begin{itemize}
            \item Hva vet vi om et element i et sortert array?
            \pause
            \item Alle elementer til venstre er mindre, alle elementer til høyre er
                større!
            \pause
            \item Når vi vet dette, hvordan kan vi utnytte det?
            \pause
            \item Sjekk midten, eliminer halve arrayet hver gang!
        \end{itemize}
    \end{frame}

    \begin{frame}
        \includegraphics[width=300pt]{sortedarray.png}
    \end{frame}


    \begin{frame}
        \frametitle{Trær}
        \begin{itemize}
            \item Datastruktur som ser ut som et opp ned tre
            \item Består av et sett med noder, som har andre noder som barn
            \item Noder kan ha en verdi, objekt, etc knyttet til seg
            \item Noen regler på hva som er lov og ikke
        \end{itemize}

        \includegraphics[keepaspectratio, width=300pt]{bst.png}

    \end{frame}

    \begin{frame}
        \frametitle{Binære søketrær}
        \begin{itemize}
            \item Binært tre med tallverdier som følger reglene til et tre
            \item Ekstra regel!
            \begin{itemize}
                \item Alle noder i høyre subtre til en node må være større
                \item Alle noder i venstre subtre til en node må være mindre
            \end{itemize}
            \item Bruker flere operasjoner på binære søketre
                \begin{itemize}
                    \item Sette inn
                    \item Slette
                    \item Finne et tall
                \end{itemize}
        \end{itemize}

    \end{frame}

    \begin{frame}
        \includegraphics[width=300pt]{bst2.png}
    \end{frame}

    \begin{frame}
        \frametitle{Mine tips for å lykkes i IN2010}
        \begin{itemize}
            \item IN2010 er et modningsfag, sett av nok tid, jobb jevnt.
            \item Visualisering er key, utnytt online ressurser for det.
            \item Mål: implementer alle algoritmene som er pensum og forstå kompleksiteten
            \item Fokuser på forståelse, ikke pugging.
            \item Diskuter med andre, hjelp andre, få hjelp av andre.
            \item Ekstra: Jobb med problemløsning ved siden av (Kattis osv)
        \end{itemize}
    \end{frame}

\end{document}

